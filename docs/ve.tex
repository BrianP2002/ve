%%%%%%%%%%%%%%%%%%%%%%%%%%%%%%%%%%%%%%%%%%%%%%%%%%%%%%%%%%%%%%%%%%%%%%%%%%%%%
%%
%A  ve.tex                      GAP documentation                Steve Linton 
%%
%A  @(#)$Id: ve.tex,v 1.1.1.1 1996/12/11 12:39:42 werner Exp $
%%
%Y  Copyright (C)  1993,  Lehrstuhl D fuer Mathematik,  RWTH Aachen,  Germany
%%
%%  This file contains the documentation of the Vector Enumeration share
%%  library.
%%
%H  $Log: ve.tex,v $
%H  Revision 1.1.1.1  1996/12/11 12:39:42  werner
%H  Preparing 3.4.4 for release
%H
%H  Revision 3.2  1994/06/17  19:04:36  vfelsch
%H  examples adjusted to version 3.4
%H
%H  Revision 3.1  1994/06/10  02:45:32  vfelsch
%H  updated examples
%H
%H  Revision 3.0  1994/05/19  13:43:58  sam
%H  Initial Revision under RCS
%H
%%
\def\VE{Vector Enumeration}
\def\MeatAxe{\sf MeatAxe}
\Chapter{Vector Enumeration}

This chapter describes the {\VE} (Version~3) share library package for
computing matrix representations of finitely presented algebras.  See
"Installing the Vector Enumeration Package" for the installation of the
package, and the {\VE} manual~\cite{Lin93} for details of the
implementation.

The default application of {\VE}, namely the function 'Operation' for
finitely presented algebras (see chapter "Finitely Presented Algebras"),
is described in "Operation for Finitely Presented Algebras".

The interface between {\GAP} and {\VE} is described in "More about Vector
Enumeration".

In "Examples of Vector Enumeration" the examples given in the {\VE} manual
serve as examples for the use of {\VE} with {\GAP}.

Finally, section "Using Vector Enumeration with the MeatAxe" shows how
the {\MeatAxe} share library (see chapter "The MeatAxe") and {\VE} can work
hand in hand.

The functions of the package can be used after loading the package with

|    gap> RequirePackage( "ve" ); |

The package is also loaded *automatically* when 'Operation' is called for the
action of a finitely presented algebra on a quotient module.

%%%%%%%%%%%%%%%%%%%%%%%%%%%%%%%%%%%%%%%%%%%%%%%%%%%%%%%%%%%%%%%%%%%%%%%%%%%%%
\Section{Operation for Finitely Presented Algebras}

'Operation( <F>, <Q> )'

This is the default application of {\VE}.
<F> is a finitely presented algebra (see chapter "Finitely Presented
Algebras"), <Q> is a quotient of a free <F>-module,
and the result is a matrix algebra representing a faithful action on <Q>.

If <Q> is the zero module then the matrices have dimension zero, so the
result is a null algebra (see "NullAlgebra") consisting only of a zero
element.

The algebra homomorphism, the isomorphic module for the matrix algebra,
and the module homomorphism can be constructed as described in chapters
"Algebras" and "Modules".

|    gap> a:= FreeAlgebra( GF(2), 2 );
    UnitalAlgebra( GF(2), [ a.1, a.2 ] )
    gap> a:= a / [ a.1^2 - a.one, # group algebra of $V_4$ over $GF(2)$
    >              a.2^2 - a.one,
    >              a.1*a.2 - a.2*a.1 ];
    UnitalAlgebra( GF(2), [ a.1, a.2 ] )
    gap> op:= Operation( a, a^1 );
    UnitalAlgebra( GF(2),
    [ [ [ 0*Z(2), 0*Z(2), Z(2)^0, 0*Z(2) ], [ 0*Z(2), 0*Z(2), 0*Z(2),
              Z(2)^0 ], [ Z(2)^0, 0*Z(2), 0*Z(2), 0*Z(2) ],
          [ 0*Z(2), Z(2)^0, 0*Z(2), 0*Z(2) ] ],
      [ [ 0*Z(2), Z(2)^0, 0*Z(2), 0*Z(2) ],
          [ Z(2)^0, 0*Z(2), 0*Z(2), 0*Z(2) ],
          [ 0*Z(2), 0*Z(2), 0*Z(2), Z(2)^0 ],
          [ 0*Z(2), 0*Z(2), Z(2)^0, 0*Z(2) ] ] ] )
    gap> Size( op );
    16 |

%%%%%%%%%%%%%%%%%%%%%%%%%%%%%%%%%%%%%%%%%%%%%%%%%%%%%%%%%%%%%%%%%%%%%%%%%%%%%
\Section{More about Vector Enumeration}
\index{VEInput}
\index{CallVE}
\index{VEOutput}
\index{FpAlgebraOps.OperationQuotientModule}

As stated in the introduction to this chapter, {\VE} is a share library
package.  The computations are done by standalone programs written in C.

The interface between {\VE} and {\GAP} consists essentially of two parts,
namely the global variable 'VE', and the function
'FpAlgebraOps.OperationQuotientModule'.

\vspace{5mm}

*The 'VE' record*

'VE' is a record with components

'Path': \\ the full path name of the directory that contains the
           executables of the standalones 'me', 'qme', 'zme',

'options': \\
  a string with command line options for {\VE};
  it will be appended to the command string of 'CallVE' (see below),
  so the default options chosen there can be overwritten.
  This may be useful for example in case of the '-v' option to enable the
  printing of comments (see section 4.3 of~\cite{Lin93}), but you should
  *not* change the output file (using '-o') when you simply call
  'Operation' for a finitely presented algebra.
  'options' is defaulted to the empty string.

\vspace{5mm}

*FpAlgebraOps.OperationQuotientModule*

This function is called automatically by 'FpAlgebraOps.Operation'
(see "Operation for Finitely Presented Algebras"),
it can also be called directly as follows.

'FpAlgebraOps.OperationQuotientModule( <A>, <Q>, <opr> )'\\
'FpAlgebraOps.OperationQuotientModule( <A>, <Q>, \"mtx\" )'

It takes a finitely presented algebra <A> and a list of submodule
generators <Q>, that is, the entries of <Q> are list of equal length,
with entries in <A>, and returns the matrix representation computed by
the {\VE} program.

The third argument must be either one of the operations 'OnPoints',
'OnRight', or the string '\"mtx\"'.  In the latter case the output will
be an algebra of {\MeatAxe} matrices, see "Using Vector Enumeration with
the MeatAxe" for further explanation.

\vspace{5mm}

*Accessible Subroutines*

The following three functions are used by
'FpAlgebraOps.OperationQuotientModule'.
They are the real interface that allows to access {\VE} from {\GAP}.

'PrintVEInput( <A>, <Q>, <names> )'

takes a finitely presented algebra <A>, a list of submodule generators
<Q>, and a list <names> of names the generators shall have in the
presentation that is passed to {\VE}, and prints a string that represents
the input presentation for {\VE}.
See section 3.1 of the {\VE} manual~\cite{Lin93} for a description of
the syntax.

|    gap> PrintVEInput( a, [ [ a.zero ] ], [ "A", "B" ] );
    2.
    A B .
    .
    .
    {1}(0).
    A*A, B*B, :
    A*B+B*A = 0, .  |

\vspace{5mm}

'CallVE( <commandstr>, <infile>, <outfile>, <options> )'

calls {\VE} with command string <commandstr>, presentation file
<infile>, and command line options <options>, and prescribes the
output file <outfile>.

If not overwritten in the string <options>, the default options
'\"-i -P -v0 -Y VE.out -L'\#  '\"' are chosen.

Of course it is not necessary that <infile> was produced using
'PrintVEInput', and also the output is independent of {\GAP}.

|    gap> PrintTo( "infile.pres",
    >             PrintVEInput( a, [ [ a.zero ] ], [ "A", "B" ] ) );
    gap> CallVE( "me", "infile", "outfile", " -G -vs2" ); |

(The option '-G' sets the output format to {\GAP}, '-vs2' chooses a
more verbose mode.)

\vspace{5mm}

'VEOutput( <A>, <Q>, <names>, <outfile> )'\\
'VEOutput( <A>, <Q>, <names>, <outfile>, \"mtx\" )'

returns the output record produced by {\VE} that was written to the file
<outfile>.
A component 'operation' is added that contains the information for the
construction of the operation homomorphisms.

The arguments <A>, <Q>, <names> describe the finitely presented algebra,
the quotient module it acts on, and the chosen generators names, i.e.,
the original structures for that {\VE} was called.

|    gap> out:= VEOutput( a, [ [ a.zero ] ], [ "A", "B" ], "outfile" );;
    gap> out.dim; out.operation.moduleinfo.preimagesBasis;
    4
    [ [ a.one ], [ a.2 ], [ a.1 ], [ a.1*a.2 ] ] |

If the optional fifth argument '\"mtx\"' is present, the output is regarded
as an algebra of {\MeatAxe} matrices (see section "Using Vector Enumeration
with the MeatAxe").  For that, an appropriate command string had to be passed
to 'CallVE'.

%%%%%%%%%%%%%%%%%%%%%%%%%%%%%%%%%%%%%%%%%%%%%%%%%%%%%%%%%%%%%%%%%%%%%%%%%%%%%
\Section{Examples of Vector Enumeration}

We consider those of the examples given in chapter 8 of the {\VE} manual
that can be used in {\GAP}.

*8.1 The natural permutation representation of $S_3$*

The symmetric group $S_3$ is also the dihedral group $D_6$, and so is
presented by two involutions with product of order 3.
Taking the permutation action on the cosets of the cyclic group generated
by one of the involutions we obtain the following presentation.

|    gap> a:= FreeAlgebra( Rationals, 2 );;
    gap> a:= a / [ a.1^2 - a.one, a.2^2 - a.one,
    >              (a.1*a.2)^3 - a.one ];
    UnitalAlgebra( Rationals, [ a.1, a.2 ] )
    gap> a.name:= "a";; |
    
We choose as module 'q' the quotient of the regular module for 'a'
by the submodule generated by 'a.1 - 1', and compute the action of 'a'
on 'q'.

|    gap> m:= a^1;;
    gap> q:= m / [ [ a.1 - a.one ] ];
    Module( a, [ [ a.one ] ] ) / [ [ -1*a.one+a.1 ] ]
    gap> op:= Operation( a, q );
    UnitalAlgebra( Rationals, 
    [ [ [ 1, 0, 0 ], [ 0, 0, 1 ], [ 0, 1, 0 ] ], 
      [ [ 0, 1, 0 ], [ 1, 0, 0 ], [ 0, 0, 1 ] ] ] )
    gap> op.name:= "op";; |
    
\vspace{5mm}

*8.2 A Quotient of a Permutation Representation*

The permutation representation constructed in example 8.1 fixes the all-ones
vector (as do all permutation representations).  This is the image of the
module element '[ a.one + a.2 + a.2\*a.1 ]' in the corresponding module for
the algebra 'op'.

|    gap> ophom:= OperationHomomorphism( a, op );;
    gap> opmod:= OperationModule( op );
    Module( op, [ [ 1, 0, 0 ], [ 0, 1, 0 ], [ 0, 0, 1 ] ] )
    gap> modhom:= OperationHomomorphism( q, opmod );;
    gap> pre:= PreImagesRepresentative( modhom, [ 1, 1, 1 ] );;
    gap> pre:= pre.representative;
    [ a.one+a.2+a.2*a.1 ] |

We could have computed such a preimage also by computing a matrix that maps
the image of the submodule generator of 'q' to the all-ones vector, and
applying a preimage to the submodule generator.  Of course the we do not
necessarily get the same representatives.
    
|    gap> images:= List( Generators( q ), x -> Image( modhom, x ) );
    [ [ 1, 0, 0 ] ]
    gap> rep:= RepresentativeOperation( op, images[1], [ 1, 1, 1 ] );
    [ [ 1, 1, 1 ], [ 1, 1, 1 ], [ 1, 1, 1 ] ]
    gap> PreImagesRepresentative( ophom, rep );
    a.one+a.1*a.2+a.2*a.1 |
    
Now we factor out the fixed submodule by enlarging the denominator of
the module 'q'.  (Note that we could also compute the action of the matrix
algebra if we were only interested in the 2-dimensional representation.)

Accordingly we can write down the following presentation for the
quotient module.

|    gap> q:= m / [ [ a.1 - a.one ], pre ];;
    gap> op:= Operation( a, q );
    UnitalAlgebra( Rationals, 
    [ [ [ 1, 0 ], [ -1, -1 ] ], [ [ 0, 1 ], [ 1, 0 ] ] ] ) |
    
\vspace{5mm}

*8.3 A Non-cyclic Module*

If we take the direct product of two copies of the permutation representation
constructed in example 8.1, we can identify the fixed vectors in the two copies
in the following presentation.

|    gap> m:= a^2;;
    gap> q:= m / [ [ a.zero, a.1 - a.one ], [ a.1 - a.one, a.zero ],
    >              [ a.one+a.2+a.2*a.1, -a.one-a.2-a.2*a.1 ] ];
    Module( a, [ [ a.one, a.zero ], [ a.zero, a.one ] ] ) / 
    [ [ a.zero, -1*a.one+a.1 ], [ -1*a.one+a.1, a.zero ], 
      [ a.one+a.2+a.2*a.1, -1*a.one+-1*a.2+-1*a.2*a.1 ] ] |
    
We compute the matrix representation.

|    gap> op:= Operation( a, q );
    UnitalAlgebra( Rationals, 
    [ [ [ 1, 0, 0, 0, 0 ], [ 0, 1, 0, 0, 0 ], [ 0, 0, 0, 1, 0 ], 
          [ 0, 0, 1, 0, 0 ], [ 1, -1, 1, 1, -1 ] ], 
      [ [ 0, 0, 1, 0, 0 ], [ 0, 0, 0, 0, 1 ], [ 1, 0, 0, 0, 0 ], 
          [ 0, 0, 0, 1, 0 ], [ 0, 1, 0, 0, 0 ] ] ] ) |
    
In this case it is interesting to look at the images of the module generators
and pre-images of the basis vectors.  Note that these preimages are elements
of a factor module, corresponding elements of the free module are again
found as representatives.

|    gap> ophom:= OperationHomomorphism( a, op );;
    gap> opmod:= OperationModule( op );;
    gap> opmod.name:= "opmod";;
    gap> modhom:= OperationHomomorphism( q, opmod );;
    gap> List( Generators( q ), x -> Image( modhom, x ) );
    [ [ 1, 0, 0, 0, 0 ], [ 0, 1, 0, 0, 0 ] ]
    gap> basis:= Basis( opmod );
    CanonicalBasis( opmod )
    gap> preim:= List( basis.vectors, x -> 
    >               PreImagesRepresentative( modhom, x ) );;
    gap> preim:= List( preim, Representative );
    [ [ a.one, a.zero ], [ a.zero, a.one ], [ a.2, a.zero ], 
      [ a.2*a.1, a.zero ], [ a.zero, a.2 ] ] |
    
\vspace{5mm}

*8.4 A Monoid Representation*

The Coxeter monoid of type $B_2$ has a transformation representation on four
points.  This can be constructed as a matrix representation over GF(3), from
the following presentation.

|    gap> a:= FreeAlgebra( GF(3), 2 );;
    gap> a:= a / [ a.1^2 - a.1, a.2^2 - a.2,
    > (a.1*a.2)^2 - (a.2*a.1)^2 ];;
    gap> q:= a^1 / [ [ a.1 - a.one ] ];;
    gap> op:= Operation( a, q );
    UnitalAlgebra( GF(3), 
    [ [ [ Z(3)^0, 0*Z(3), 0*Z(3), 0*Z(3) ], [ 0*Z(3), 0*Z(3), Z(3)^0,
              0*Z(3) ], [ 0*Z(3), 0*Z(3), Z(3)^0, 0*Z(3) ],
          [ 0*Z(3), 0*Z(3), 0*Z(3), Z(3)^0 ] ],
      [ [ 0*Z(3), Z(3)^0, 0*Z(3), 0*Z(3) ],
          [ 0*Z(3), Z(3)^0, 0*Z(3), 0*Z(3) ],
          [ 0*Z(3), 0*Z(3), 0*Z(3), Z(3)^0 ],
          [ 0*Z(3), 0*Z(3), 0*Z(3), Z(3)^0 ] ] ] ) |
    
\vspace{5mm}

*8.7 A Quotient of a Polynomial Ring*

The quotient of a polynomial ring by the ideal generated by some polynomials
will be finite-dimensional just when the polynomials have finitely many common
roots in the algebraic closure of the ground ring.
For example, three polynomials in three variables give us the following
presentation for the quotient of their ideal.

Define 'a' to be the polynomial algebra on three variables.

|    gap> a:= FreeAlgebra( Rationals, 3 );;
    gap> a:= a / [ a.1 * a.2 - a.2 * a.1,
    >              a.1 * a.3 - a.3 * a.1,
    >              a.2 * a.3 - a.3 * a.2 ];; |
    
Define the quotient module by the polynomials 'A+B+C', 'AB+BC+CA', 'ABC-1'.

|    gap> q:= a^1 / [ [ a.1+a.2+a.3 ],
    >                [ a.1*a.2+a.2*a.3+a.3*a.1 ],
    >                [ a.1*a.2*a.3-a.one ]        ];; |
    
Compute the representation.

|    gap> op:= Operation( a, q );
    UnitalAlgebra( Rationals, 
    [ [ [ 0, 1, 0, 0, 0, 0 ], [ 0, 0, 0, 0, 1, 0 ],
          [ -1, 0, 0, 0, 0, -1 ], [ 0, 0, 1, 0, 0, 0 ],
          [ 1, 0, 0, 0, 0, 0 ], [ 0, -1, 0, -1, 0, 0 ] ],
      [ [ 0, 0, 0, 1, 0, 0 ], [ 0, 0, 1, 0, 0, 0 ], [ 0, 0, 0, 0, 0, 1 ],
          [ 0, 0, -1, 0, -1, 0 ], [ -1, 0, 0, 0, 0, -1 ],
          [ 0, 1, 0, 0, 0, 0 ] ],
      [ [ 0, -1, 0, -1, 0, 0 ], [ 0, 0, -1, 0, -1, 0 ],
          [ 1, 0, 0, 0, 0, 0 ], [ 0, 0, 0, 0, 1, 0 ],
          [ 0, 0, 0, 0, 0, 1 ], [ 0, 0, 0, 1, 0, 0 ] ] ] ) |
    
%%%%%%%%%%%%%%%%%%%%%%%%%%%%%%%%%%%%%%%%%%%%%%%%%%%%%%%%%%%%%%%%%%%%%%%%%%%%%
\Section{Using Vector Enumeration with the MeatAxe}

One can deal with the matrix representation constructed by {\VE} also using
the {\MeatAxe} share library.  This way the matrices are not read into {\GAP}
but written to files and converted into internal {\MeatAxe} format.
See chapter "The MeatAxe" for details.

|    gap> a:= FreeAlgebra( GF(2), 2 );;
    gap> a:= a / [ a.1^2 - a.one, a.2^2 - a.one,
    >              (a.1*a.2)^3 - a.one ];;
    gap> RequirePackage("meataxe");
    &I  The MeatAxe share library functions are available now.
    &I  All files will be placed in the directory
    &I     '/var/tmp/tmp.006103'
    &I  Use 'MeatAxe.SetDirectory( <path> )' if you want to change.
    gap> op:= Operation( a, a^1, "mtx" );
    UnitalAlgebra( GF(2), 
    [ MeatAxeMat( "/var/tmp/tmp.006103/a/g.1", GF(2), [ 6, 6 ], a.1 ),
      MeatAxeMat( "/var/tmp/tmp.006103/a/g.2", GF(2), [ 6, 6 ], a.2 ) ] )
    gap> Display( op.1 );
    &I  calling 'maketab' for field of size 2
    MeatAxe.Matrix := [
    [0,0,1,0,0,0],
    [0,0,0,1,0,0],
    [1,0,0,0,0,0],
    [0,1,0,0,0,0],
    [0,0,0,0,0,1],
    [0,0,0,0,1,0]
    ]*Z(2);
    gap> MeatAxe.Unbind(); |

%%%%%%%%%%%%%%%%%%%%%%%%%%%%%%%%%%%%%%%%%%%%%%%%%%%%%%%%%%%%%%%%%%%%%%%%%%%%%
%%
%E  Emacs . . . . . . . . . . . . . . . . . . . . . . . local emacs variables
%%
%%  Local Variables:
%%  mode:               outline
%%  outline-regexp:     "%F\\|%V\\|%E"
%%  fill-column:        73
%%  fill-prefix:        "%%  "
%%  eval:               (hide-body)
%%  End:
